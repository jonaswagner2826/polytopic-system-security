\section{Basic Definitions}
\subsection{General Sets / Nomenclature}
Basically a \textbf{\emph{set}} is the collection of things. 

A set is \textbf{\emph{Bounded}} if ...\[
    \exists_{M} \st \forall_{x \in X} x \leq M
\]

A set is \textbf{\emph{closed}} if (operations on members of a class results in another member of the class) \[
    X \text{closed under} f(\cdot,\cdot,\dots) \iff \forall_{x,y} \in X \implies f(x, y, \dots) \in X
\]

The space $\R$ is ... 
The space $R^n$ is ...


All the other definitions... 

\subsubsection{Set Operations}
\begin{definition}
    Let $Z, W \subset \R^n$, $Y \subset \R^k$, and $\vb{R} \subset R^{k \cross n}$.
    \begin{enumerate}
        \item A \textbf{Linear Mapping} of $Z$ defined as \begin{equation}
            \vb{R} Z \equiv \{\vb{R} \vb{z} \st \vb{z} \in Z\}
        \end{equation}
        \item A \textbf{Minkowski Sum} of $Z$ and $W$ is defined as \begin{equation}
            Z + W \equiv \{\vb{z} + \vb{w} \st \vb{z} \in Z, \vb{w} \in W\}
        \end{equation}
        \item A \textbf{Generalized Intersection} of $Z$ and $Y$ is defined as \begin{equation}
            Z \cap_{\vb{R}} Y \equiv \{\vb{z} \in Z \st \vb{R} \vb{z} \in Y\}
        \end{equation}
        which is a standard intersection $\cap$ for $k = n$ and $\vb{R} = \vb{I}_{n \cross n}$.
    \end{enumerate}

\end{definition}

\subsection{Specific Set Definitions}
\subsubsection{Convex Polytope}
\begin{definition}
    $P \subset \R^n$ is a \emph{\textbf{Convex Polytope}} if it is Bounded and \begin{equation} \label{eq:convex_polytope_def}
        \exists (\vb{H}, \vb{k}) \in \R^{n_h \cross n} \cross \R^{n_h} \st P = \{\vb{z} \in \R^n \st \vb{H} \vb{z} \leq \vb{k}\}
    \end{equation}
\end{definition}

\textbf{Notes:}
\begin{itemize}
    \item \eqref{eq:convex_polytope_def} is known as a \emph{halfspace-representation} (H-rep) of $P$.
    \footnote{I've also known this as an Affine version of a polytope as opposed to the standard convex hull definition.} 
    \item A polytope can also be represented as the convex hull of the vertices (V-rep).
\end{itemize}

\subsubsection{Zonotope}
\begin{definition}
    $Z \subset \R^n$ is \emph{\textbf{Zonotope}} if \begin{equation}\label{eq:zonotope_def}
        \exists (\vb{G}, \vb{c}) \in \R^{n \cross n_g} \cross \R^{n} \st Z = \{\vb{G} \xi + \vb{c} \st \norm{\xi}_\infty \leq 1\}
    \end{equation}
\end{definition}

\textbf{Notes:}
\begin{itemize}
    \item $Z$ defined by \eqref{eq:zonotope_def} can be denoted by $Z = \{\vb{G}, \vb{c}\}$.
    \item \eqref{eq:zonotope_def} is known as the \emph{generator-representation} (G-rep) where $\vb{c}$ is called the \emph{center} and the columns of $\vb{G}$ are the \emph{generators}.
    \item The \emph{\textbf{order}} of a Zonotope is $n_g / n$.
\end{itemize}

\textbf{Special Zonotopes:}
\begin{enumerate}
    \item $Z$ is a \emph{\textbf{parallelotope}} if $Z$ is a zonotope with $n_g = n$.
    \item $Z$ is an \emph{\textbf{interval}} if $\vb{G} = \vb{I}_{n \cross n}$.
\end{enumerate}

\textbf{Properties:}
\begin{enumerate}
    \item Zonotopes are \emph{\textbf{centrally symentric}}
    (i.e. every chord through $\vb{c}$ is bisected by $\vb{c}$).\[
        \text{A convex polytope is a zonotope} \iff \text{every 2-face is centrally symmetric}
    \]
    \item All Zonotopes are affine image of the $\infty$-norm unit ball.
\end{enumerate}

\textbf{Operations:}
Let $Z = \{\vb{G}_z, \vb{c}_z\}$ and $W = \{\vb{G}_w, \vb{c}_w\}$.
\begin{align}
    \vb{R} Z &= \{\vb{R} \vb{G}_z, \vb{R} \vb{c}_z\}\\
    Z + W    &= \{\mqty[\vb{G}_z & \vb{G}_w], \vb{c}_z + \vb{c}_w\}
\end{align}

\subsubsection{Ellipsoid}
\begin{definition} \label{eq:ellipsoid_def}
    $E \subset \R^n$ is an \emph{\textbf{Ellipsoid}} if \begin{equation}
        \exists (\vb{Q}, \vb{c}) \in \R^{n \cross n} \cross \R^{n} \st E = \{\vb{Q} \xi + \vb{c} \st \norm{\xi}_2 \leq 1\}
    \end{equation}
\end{definition}

\textbf{Notes:}
\begin{itemize}
    \item \eqref{eq:ellipsoid_def} represents the degenerate ellipsoid when $\vb{Q}$ is singular.
    \item If $\vb{Q}$ is invertable, then \eqref{eq:ellipsoid_def} is equivelent to \[
        E = \qty{\vb{z} \st (\vb{z} - \vb{c})^T \qty(\vb{Q} \vb{Q}^T)^{-1} (\vb{z} - \vb{c}) \leq 1}
    \]
    \item We denote shorthand for the ellipsoid $E$ defined in \eqref{eq:ellipsoid_def} as $E = \{\vb{Q}, \vb{c}\}$ where $Q$ is known as the covariance matrix and $c$ is the center.
\end{itemize}

\textbf{Properties:}
\begin{enumerate}
    \item All Ellipsoids are affine image of the $2$-norm unit ball.
\end{enumerate}